% Modelo benchmark tradicional
\section{Metodología}

\begin{frame}[t,plain]
	\vspace{0.4cm}
	\begin{flushleft}
		{\Large\bfseries Modelo benchmark tradicional}
	\end{flushleft}

	\vspace{0.6cm}
	Se propone un modelo \textbf{probit ordenado con ceros inflados}, que distingue dos decisiones:

	\vspace{0.4cm}
	\begin{itemize}
		\item \textbf{Participación (consumir o no):} vinculada a barreras institucionales, normas sociales o riesgos percibidos.

		\item \textbf{Intensidad del consumo (ocasional o frecuente):} entre quienes ya consumen.
	\end{itemize}

	\vspace{0.4cm}
	Este enfoque es coherente con la teoría económica al separar decisiones que tienen fundamentos distintos. También sirve como base comparativa frente a modelos más flexibles, como los de machine learning.

	\vfill
\end{frame}


