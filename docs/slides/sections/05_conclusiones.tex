% Conclusiones
\section{Conclusiones}

\begin{frame}
\frametitle{Hallazgos Clave}
\begin{itemize}
    \item El modelo integra datos de 169,346 individuos con características sociodemográficas
    \item Las variables demográficas son determinantes en la predicción de ingresos tributarios
    \item Diferentes marcos regulatorios generan variaciones significativas en recaudos
    \item La formalización de este mercado tiene potencial fiscal importante para Colombia
\end{itemize}
\end{frame}

\begin{frame}
\frametitle{Implicaciones de Política Pública}
\begin{enumerate}
    \item \textbf{Diseño tributario}: Estructura impositiva óptima según escenarios
    \item \textbf{Regulación}: Marcos normativos que maximicen recaudos y control
    \item \textbf{Segmentación}: Políticas diferenciadas por perfil sociodemográfico
    \item \textbf{Fiscalización}: Mecanismos efectivos de cumplimiento tributario
\end{enumerate}
\end{frame}

\begin{frame}
\frametitle{Próximos Pasos}
\begin{itemize}
    \item Refinar modelos predictivos con variables económicas adicionales
    \item Incorporar elasticidades precio de demanda por segmento
    \item Análisis de spillovers con mercados ilegales
    \item Validación de proyecciones con casos internacionales
    \item Actualización con datos más recientes
\end{itemize}
\end{frame}

\begin{frame}[plain]
\centering
\Huge \textbf{Preguntas y Discusión}
\end{frame}
