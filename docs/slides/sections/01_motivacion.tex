% Introducción y Motivación
\section{Introducción}

\begin{frame}
\frametitle{Contexto Global}
\begin{itemize}
    \item \textbf{Tendencia mundial}: Más de 40 países han legalizado cannabis en alguna forma
    \item \textbf{Impacto fiscal}: Generación de ingresos tributarios significativos en jurisdicciones que regulan
    \item \textbf{Colombia}: Contexto único con historial en economías ilícitas de drogas
    \item \textbf{Oportunidad}: Regulación de mercados alternativos para ingresos públicos
\end{itemize}
\end{frame}

\begin{frame}
\frametitle{Brecha de Conocimiento}
\begin{itemize}
    \item Escasez de modelos predictivos sobre estructura tributaria en mercados legalizados
    \item Falta de análisis cuantitativos específicos para Colombia
    \item Necesidad de herramientas para proyecciones fiscales bajo distintos marcos regulatorios
    \item Integración de datos sociodemográficos para segmentación de mercado
\end{itemize}
\end{frame}

\begin{frame}
\frametitle{¿Por Qué Importa?}
\begin{columns}
\column{0.5\textwidth}
\textbf{Dimensión Fiscal:}
\begin{itemize}
    \item Ingresos tributarios potenciales
    \item Presión sobre finanzas públicas
    \item Eficiencia recaudatoria
\end{itemize}

\column{0.5\textwidth}
\textbf{Dimensión Social:}
\begin{itemize}
    \item Formalización económica
    \item Reducción de economía ilícita
    \item Regulación y control
\end{itemize}
\end{columns}
\end{frame}
