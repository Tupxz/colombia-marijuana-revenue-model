% Metodología
\section{Metodología}

\begin{frame}
\frametitle{Datos y Fuentes}
\begin{columns}[T]
\column{0.48\textwidth}
\textbf{Base Disponible}
\begin{itemize}
\item 169,346 individuos
\item Encuesta Nacional (DANE)
\item 12 variables sociodemográficas
\item Múltiples capítulos por explorar
\end{itemize}

\column{0.48\textwidth}
\textbf{Por Integrar}
\begin{itemize}
\item DIAN (recaudos)
\item DANE (macro)
\item Capítulos adicionales
\item Análisis exploratorio
\end{itemize}
\end{columns}
\end{frame}

\begin{frame}
\frametitle{Variable Objetivo}

\textbf{Unidad de Análisis}: Colombia en períodos anuales/semestrales

\textbf{Definición}:
$$Y_{k,t} = \frac{\text{Recaudo fuente } k}{\text{Recaudo total}}$$

\textbf{Características}:
\begin{itemize}
\item \textbf{Composicional}: Las participaciones suman 1
\item \textbf{Acotada}: $Y_{k,t} \in [0,1]$ para cada fuente
\item \textbf{Escenario-dependiente}: Impuesto a cannabis solo en legalización
\end{itemize}
\end{frame}

\begin{frame}
\frametitle{Estrategia Analítica}

\textbf{Principio rector}: Primero conocer la naturaleza de los datos

\begin{enumerate}
\item \textbf{Exploración}: Descubrimiento y selección de variables
\item \textbf{Procesamiento}: Limpieza y normalización
\item \textbf{Análisis descriptivo}: Patrones demográficos
\item \textbf{Modelado}: Machine Learning + Benchmark probit inflado
\item \textbf{Simulación}: Proyecciones bajo escenarios
\end{enumerate}

\vspace{0.2cm}
\textit{Flexibilidad}: Modelo específico ML a especificar según hallazgos
\end{frame}

\begin{frame}
\frametitle{Modelo Benchmark}

\textbf{Logit Fraccionado}

\begin{itemize}
\item Maneja datos acotados en [0,1]
\item Captura características de datos composicionales
\item Referencia para evaluar modelos de ML
\item Interpretabilidad probabilística clara
\end{itemize}

\vspace{0.3cm}
\textit{Especificación}: $E[Y_{k,t}|X] = \frac{\exp(\eta)}{1 + \exp(\eta)}$
\end{frame}

\begin{frame}
\frametitle{Supuestos Fundamentales}
\begin{enumerate}
\item \textbf{Shock estructural}: Legalización aumenta transacciones formales

\item \textbf{Movilidad de demanda}: Mercado legal afectado por regulación y precios

\item \textbf{Fuente tributaria}: Impuesto cannabis es identificable y persistente

\item \textbf{Transmisión de efectos}: Cambios en consumo se reflejan en estructura de recaudos
\end{enumerate}
\end{frame}

\begin{frame}
\frametitle{Escenarios de Simulación}

\begin{columns}[T]
\column{0.5\textwidth}
\textbf{Escenario Base}
Prohibición mantenida

\vspace{0.3cm}
\textbf{Escenario 1}
Legalización + IVA 19\%

\column{0.5\textwidth}
\textbf{Escenario 2}
Legalización + Impuesto 25\%

\vspace{0.3cm}
\textbf{Escenario 3}
Regulación parcial + Tramos múltiples
\end{columns}

\vspace{0.3cm}
\textit{Variables de sensibilidad}: Elasticidades, tasas de penetración
\end{frame}

\begin{frame}
\frametitle{Evaluación del Modelo}

\begin{itemize}
\item \textbf{MAE/RMSE}: Error medio absoluto y cuadrático
\item \textbf{Validación temporal}: Desempeño fuera de muestra
\item \textbf{Coherencia composicional}: Participaciones $\sum Y_k = 1$
\item \textbf{Robustez}: Estabilidad ante cambios de especificación
\end{itemize}
\end{frame}
