\documentclass[12pt, a4paper]{article}

\usepackage[spanish]{babel}
\usepackage[utf8]{inputenc}
\usepackage[T1]{fontenc}
\usepackage{setspace}
\usepackage[margin=1in]{geometry}
\usepackage{amsmath}
\usepackage{amssymb}
\usepackage{graphicx}
\usepackage{hyperref}
\usepackage{natbib}
\usepackage{abstract}

\onehalfspacing

\title{Modelado de Recaudos Tributarios en Mercados Legalizados: Un Análisis Predictivo de la Distribución de Cannabis en Colombia}

\author{Santiago Tupaz$^{1,2}$ \and Moisés Quintero$^{1}$ \and Vanessa Rodrigues$^{1}$ \\[0.5cm] 
{\small $^{1}$ Universidad EAFIT, Programa de Economía, Medellín, Colombia} \\
{\small $^{2}$ Universidad Nacional de Colombia, Programa de Estadística, Sede Medellín} \\[1cm]
{\small \textbf{Profesora Supervisora:} Paula María Almonacid Hurtado, Universidad EAFIT}}

\date{\small Febrero 9, 2026}

\begin{document}

\begin{titlepage}
    \centering
    \vspace*{3cm}

    {\Large \textbf{Universidad EAFIT}}\\[0.3cm]
    {\normalsize Programa de Economía}\\
    {\normalsize Ciencia de Datos}\\[2cm]

    {\large \textbf{Modelado de Recaudos Tributarios en Mercados Legalizados}}\\[0.5cm]
    {\small Un Análisis Predictivo de la Distribución de Cannabis en Colombia}\\[2cm]

    {\normalsize
    \textbf{Autores:}\\
    Santiago Tupaz$^{1,2}$\\
    Moisés Quintero$^{1}$\\
    Vanessa Rodrigues$^{1}$
    }\\[1.5cm]

    {\small
    $^{1}$ Universidad EAFIT, Programa de Economía, Medellín, Colombia\\[0.3cm]
    $^{2}$ Universidad Nacional de Colombia, Programa de Estadística, Sede Medellín
    }\\[1.5cm]

    {\normalsize
    \textbf{Profesora Supervisora:}\\
    Paula María Almonacid Hurtado\\
    {\small Universidad EAFIT}
    }\\[2.5cm]

    {\normalsize
    Medellín, Colombia\\
    9 de febrero de 2026
    }

    \vfill
\end{titlepage}

\begin{abstract}
Este artículo examina cómo podría proyectarse la estructura del recaudo tributario del Estado colombiano bajo distintos escenarios de legalización y regulación de la distribución de marihuana. Utilizando técnicas de ciencia de datos y análisis econométrico, construimos un modelo predictivo que integra variables sociodemográficas, datos de encuestas nacionales y proyecciones fiscales. Los resultados proporcionan evidencia cuantitativa para la formulación de políticas públicas en el contexto de la reforma tributaria y regulación de mercados alternativos.

\textit{Palabras clave}: recaudo tributario, legalización de drogas, predicción, ciencia de datos, política fiscal
\end{abstract}

\newpage

\tableofcontents

\newpage

\section{Introducción}

La legalización de sustancias controladas representa una transformación significativa en la política pública, con implicaciones fiscales, sociales y económicas profundas. El caso de la marihuana en Colombia ofrece un contexto particularmente relevante, dado el historial del país en relación con economías ilegales relacionadas con drogas y la creciente consideración de marcos regulatorios alternativos.

Este estudio aborda una pregunta fundamental para la formulación de políticas públicas: \textit{¿Cómo podría estructurarse el recaudo tributario del Estado colombiano bajo distintos escenarios de legalización de la distribución de marihuana?}

Nuestro enfoque combina técnicas de ciencia de datos, análisis econométrico y modelado de escenarios para proporcionar proyecciones cuantitativas que contribuyan al debate de política fiscal y regulatoria. El análisis integra datos sociodemográficos, información de encuestas nacionales y variables macroeconómicas relevantes.

\subsection{Motivación}

\subsection{Objetivos}

\subsection{Estructura del documento}

\end{document}
